A \textbf{(Commutative) Ring} $R$ is a set with two binary operation $+$ and $\times$,
refered to as \textbf{Addition} and \textbf{Multiplication}, such that
\begin{enumerate}
    \item $(R, +)$ is an abelian group denoted as $R^+$, with the identity denoted as $0$.
    \item $\times$ is Commutative and Associative, with the identity denoted as $1$.
    \item $\lall a,b,c\in R, (a+b)c = ac + bc$. (Distributive Law)
\end{enumerate}

A \textbf{Noncommutative Ring} is a Ring without the Commutative Law for Multiplication.

A \textbf{Subring} of $R$ is a Ring that is a subset of $R$.

A \textbf{Ring Homomorphism} $\bm{\phi}$ is a function from $R \to R'$
such that $\lall a,b\in R$,
\begin{enumerate}
    \item $\phi(a+b) = \phi(a)+\phi(b)$.
    \item $\phi(ab) = \phi(a)\phi(b)$.
    \item $\phi(1)=1$.
\end{enumerate}

$R$ and $R'$ are \textbf{Isomorphic}, denoted by $R \approx R'$ 
$\iff$
There exists an Ring Isomorphism between them.

An \textbf{Ideal} $\bm{I}$ of a ring $R$ is a subset of $R$ such that
\begin{enumerate}
    \item $I \neq \emptyset$.
    \item $I$ is closed under addition.
    \item $\lall s \in I, \lall r \in R$, $rs \in I$.
\end{enumerate}

The kernel of a ring homomorphism is an Ideal.

Let $a\in \R$.
The multiples of $a$ form an Ideal called the \textbf{Principal Ideal} generated by $a$.
Denoted as $(a) = aR = Ra = \set{ra | r \in R}$.

$(1) = R$ is the \textbf{Unit Ideal}.

$(0) = \set{0}$ is the \textbf{Zero Ideal}.

A Ideal that is not $(1)$ nor $(0)$ is \textbf{Proper}.

A Ring with exactly two ideals is a Field.

An Ideal generated by a set of elements $\set{a_1, \hdots, a_n} \subseteq R$ is the smallest
ideal that contains those elements. 
Denoted as $(a_1, \hdots, a_n) = \set{r_1a_1 + \hdots + r_na_n | r_i \in R}$.