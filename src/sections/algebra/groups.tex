% Let $S$ be a set. Suppose $a,b,c \in S$.

A \textbf{Law of Composition/(Closed) Binary Operation on} $S$
% maps any ordered pair of $S$ to some element is $S$. In other words, 
is a function/map of the form
$S \times S \to S$. \\
(Common Notations: $*, \star, \cdot, \dotp, \bullet, \times, +$, or by concatenating the two elements.)

A Law of Composition on $S$ is 
\(
    \begin{cases}
        \textbf{Associative} &\iff \lall a,b,c \in S, (ab)c = a(bc). \\
        \textbf{Commutative} &\iff \lall a,b \in S, ab = ba.
    \end{cases}
\)

The \textbf{Identity/Unity} of a Law of Composition is an element $\bm{e} \in S$,
such that $\lall a \in S, ea = ae = a$.

A element $a \in S$ is \textbf{Invertible} $\iff \lis b \in S, ab = ba = e$, 
where $b$ is the inverse of $a$.

A \textbf{Group} is a set $S$ together with a Law of Composition
denoted by $\bm{G} = (S, *)$, such that
\begin{enumerate}
    \item * is Associative.
    \item $\lis$ an Identity.
    \item Every element of $S$ is Invertible.
\end{enumerate}

A \textbf{Abelian Group} is a Group that is Commutative.

The \textbf{Order} of a Group $(S, *)$ is the number of elements in the group,
where $\bm{\abs{G}} = \abs{(S, *)} = \abs{S}$. \\
(Notation: $(S, *)$ should not be treated as an ordered set, 
for instance: $(\R, \times) = \R^\times$, and $(\Z, +) \subseteq (\R, +)$.)

The \textbf{Center} of a group $G$ is defined: 
$\bm{Z} = \set{z \in G | \lall x \in G, zx = xz}$.

A Group $H$ is a \textbf{Subgroup of} a Group $G$
$\iff H \subseteq G \land \lall a,b \in H, ab \in H$ (\textbf{Closure}). \\
(Closure is a part of the definition of a Law of Composition.)

Every Group $G$ has two trivial Subgroups: $G$ and $\set{e}$.

A \textbf{Cyclic Subgroup} $H$ of $G$ is generated by a element $x \in G$,
such that $H = \set{x^n | n \in \Z}$.

$H$ is a \textbf{Normal Subgroup} of $G$
$\iff \lall h \in H, \lall g \in G, ghg^{-1} \in H$. 





Let $G, G'$ be groups.

A \textbf{Homomorphism} $\bm{\phi}$ is a function from $G \to G'$,
such that $\lall a,b \in G, \phi(ab) = \phi(a) \phi(b)$.

The \textbf{Kernel} $\bm{\textbf{\kernel}\phi}$ of a Homomorphism is the set of all elements in $G$
such that $a \in G, \phi(a) = $ the identity in $G'$.