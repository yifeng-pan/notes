% Let $S$ be a set. Suppose $a,b,c \in S$.

A \textbf{Law of Composition/(Closed) Binary Operation on} $S$
% maps any ordered pair of $S$ to some element is $S$. In other words, 
is a function/map of the form
$S \times S \to S$. \\
(Common Notations: $*, \star, \cdot, \dotp, \bullet, \times, +$, or by concatenating the two elements.)

A Law of Composition on $S$ is 
\(
    \begin{cases}
        \textbf{Associative} &\iff \lall a,b,c \in S, (ab)c = a(bc). \\
        \textbf{Commutative} &\iff \lall a,b \in S, ab = ba.
    \end{cases}
\)

The \textbf{Identity/Unity} of a Law of Composition is an element $\bm{e} \in S$,
such that $\lall a \in S, ea = ae = a$.

A element $a \in S$ is \textbf{Invertible} $\iff \lis b \in S, ab = ba = e$, 
where $b$ is the inverse of $a$.

A \textbf{Group} is a set $S$ together with a Law of Composition
denoted by $\bm{G} = (S, *)$, such that
\begin{enumerate}
    \item * is Associative.
    \item $\lis$ an Identity $\bm{e_G}$.
    \item Every element of $S$ is Invertible.
\end{enumerate}

A \textbf{Abelian Group} is a Group that is Commutative.

The \textbf{Order} of a Group $(S, *)$ is the number of elements in the group,
where $\bm{\abs{G}} = \abs{(S, *)} = \abs{S}$. \\
(Notation: $(S, *)$ should not be treated as an ordered set, 
for instance: $(\R, \times) = \R^\times$, and $(\Z, +) \subseteq (\R, +)$.)

The \textbf{Center} of a group $G$ is defined: 
$\bm{Z} = \set{z \in G | \lall x \in G, zx = xz}$.


A \textbf{Partition} $\Pi$ of a set $S$ is a subdivision of $S$ into nonoverlapping, nonempty subsets.

A Partition is logically equivalent to an \textbf{Equivalence Relation} on $S$.


\subsection{Subgroups}
    A Group $H$ is a \textbf{Subgroup of} a Group $G$
    $\iff H \subseteq G \land \lall a,b \in H, ab \in H$ (\textbf{Closure}). \\
    (Closure is a part of the definition of a Law of Composition.)

    Every Group $G$ has two trivial Subgroups: $G$ and $\set{e}$ (The trivial subgroups are also Normal).

    A \textbf{Cyclic Subgroup} $H$ of $G$ is generated by a element $x \in G$,
    such that $H = \set{x^n | n \in \Z}$.

    $H$ is a \textbf{Normal Subgroup} of $G$
    $\iff \lall h \in H, \lall g \in G, ghg^{-1} \in H$. 


\subsection{Homomorphisms}
    Let $G, G'$ be groups.

    A \textbf{Homomorphism} $\bm{\phi}$ is a function from $G \to G'$,
    such that $\lall a,b \in G, \phi(ab) = \phi(a) \phi(b)$.

    The \textbf{Kernel} $\bm{\textbf{\kernel}(\phi)}$ of a Homomorphism is the set of all elements in $G$
    such that $a \in G, \phi(a) = $ the identity in $G'$.

    An \textbf{Isomorphism} $\psi: G \to G'$ is a bijective group Homomorphism.


\subsection{Cosets}
    Let $H$ be a subgroup of $G$, $a \in G$.\\
    $\bm{aH = \set{ah | h \in H}}$ is a \textbf{Left Coset}. \\
    $\bm{Ha = \set{ha | h \in H}}$ is a \textbf{Right Coset}.

    All Left Cosets of a subgroup $H$ of $G$ have the same order,
    and form a Partition of $G$.

    The \textbf{Index} of $H$ in $G$ is the number of left cosets of the subgroup.
    Denoted by $\abs{G:H}$.

    Right Cosets have the same properties, but may not equal to the Left Coset.

    Let $\psi:G \to G'$ be a group homorphism.
    $\abs{G:\ker{\psi}} = \abs{\psi(G)}$.


\subsection{Quotient Group}
    Let $C$ be the set of left cosets of a Normal subgroup $N$ of $G$.

    A \textbf{Quotient Group} $\bm{G/N}$ is defined on $C$ such that: 
    $\lall aN,bN \in G/N, (aN) (bN) = abN$.


\subsection{Symmetric Group}
    A \textbf{Symmetric group} $\bm{S_n}$ is the set of all permutations of $\set{1, 2, ... , n}$.

    % Let $p$ be a permutation.

    % There exists a \textbf{Permutation Matrix} $P$ for $p$

    Every permutation $p$ has an associated \textbf{Permutation Matrx} $P$ 
    with a single $1$ in each row and column, and $0$s everywhere else.

    $P^{-1} = P^T$.

    A permutation $p$ is \textbf{Even} $\iff \det(P) = 1$.\\
    A permutation $p$ is \textbf{Odd} $\iff \det(P) = -1$.

    An \textbf{Alternating Group} $\bm{A_n}$ contains all even elements of $S_n$.

    The order of $p$ is the smallest positive integer $k$ such that $p^k = 1$.

\subsection{Dihedral Group TODO}

\subsection{Action/Operation}
    Let $G$ be a group, $X$ be a set.

    A \textbf{Let Group Action (Operation)} is a function 
    $\bm{\phi}: G\times X \to X, (g,x) \mapsto \phi(g,x) = g \dotp x$
    such that 
    \begin{enumerate}
        \item $\lall s \in S, \phi(e_G, x) = x$.
        \item $\lall g, h \in G, \lall x \in X, \phi(gh,x) = \phi(g,\phi(h,x))$.
    \end{enumerate}

    A \textbf{Right Group Action} is defined similarly.

    A \textbf{Orbit} of an element $x \in X$ by the group $G$ 
    is $G\dotp x = \set{g \dotp x | g \in G}$.

    The set of Orbits of $X$ form a partition of $X$.

    A \textbf{Stabilizer} of $x$ by $G$ is $g\in G | g \dotp x = x$.\\
    The set of all Stabilizers is the \textbf{Stabilizer (Subgroup)} of $x$ under $G$.

\subsection{Conjugation}
    A \textbf{Conjugation} is a action of $G$ on itself such that $(g,x) \mapsto gxg^{-1}$.

    The Stabilizer of this action is called the \textbf{Centralizer} denoted by $\bm{Z(x)}$.

    A Orbit of this action is called the \textbf{Conjugacy Class} doneted by $\bm{C(x)}$.

    $\abs{G} = \abs{Z(x)}\abs{C(x)}$.